%!TEX root = main.tex
\section{High-Level Description}
Conversation understanding has been the subject of 
extensive research \cite{...}. These works explore 
different aspects of conversations, from winning an argument 
to automatically recognize the speaker at each segment of the dialog. 
Our project aims to employ NLP and machine learning techniques to 
understand the {\em bottom-line} of a conversation, based on segments 
of it. Such a system can be highly useful in the context of automatic 
chatbots which can asses the probability of a purchase at the end of 
the conversation or what strategy it should employ in order to get there. 
Customer service teams can also benefit from the system as they can decide 
whether a customer is calling to terminate the contract with the company 
or just to get a discount. 
\amir{Amir to detail more}

\section{Solution Details}
To predict the bottom-line, our solution will include several stages. 
First, we will identify the topic of the conversation and named entities 
appearing in it. Second, we will segment the conversation, 
and cluster the different segments to identify types of 
``Conversational building blocks'' and learn probabilities 
of transferring from one type of segment to another. 
On top of the segments classification, we will extract features of 
conversation (e.g., dominance of one of the parties, 
the volume of text in the replies, usage of positive/negative words, 
time lapses between responses). 
We will use these features as input to MEMM/RNN to predict the 
most probable continuation of the conversation 
(similarly to what is done with Part-of-Speech tagging). 
\amir{Slava to detail about strategy}


\section{Data}
We plan to make use of all or a part of the data sources detailed here. 
\begin{itemize}
	\item {\bf The NPS Chat Corpus:} part of the Natural Language Toolkit (NLTK) distribution. 
	NLTK is a ``platform for building Python programs to work with human language data''. 
	It includes both the whole NPS Chat Corpus, as well as a number of modules for working with the data. 
	Each sentence includes POS tags and a class (e.g., question) \cite{NPS}.

	\item {\bf Kaggle compatitions: }

	\item {\bf Camidal.or/~mh521/dstc}
\end{itemize}
\amir{Efrat to continue detail about data}
