%!TEX root = main.tex
\section{Network Architecture}\label{sec:network}
For the architecture of our network , 
Our network (whose architecture is depicted in Figure \ref{fig:arch}) consists of two main components: 
a text encoder for which we have augmented the hierarchical attention network described in \cite{attention}, 
and a semantic feature encoder. 

\subsection{Text Encoder}

\paragraph*{GRU Encoder}
We have divided the task of encoding the dialog into several 
layers of encoders, each dealing with the dialog in a different granularity. 
At the lowest lever, we encode the words of each dialog into vectors 
and use word-level attention. Then a sentence encoder and sentence level attention 
translate the sentences into vectors, again using sentence-level attention, to extract a vector describing the text. 
We use a GRU-based sequence encoder \cite{BahdanauCB14} which uses 
a reset gate and an update gate ($r_t$ and $z_t$, respectively), 
so that at time $t$ the state of the GRU is 

\begin{equation*}
	h_t = (1-z_t)\cdot h_{t-1} + z_t \cdot \widetilde{h_t}
\end{equation*}

Where $h_{t-1}$ is the previous state and $\widetilde{h_t}$ 
is the current new state, computed with the new sequence information. 
$z_t$ controls the amount of past information that is kept 
and the amount of newly added information. $z_t$ is updated like so

\begin{equation*}
	z_t = \sigma(W_z x_t + U_z h_{t-1} + b_z)
\end{equation*}

$x_t$ is the sequence vector at time $t$ and 
$\widetilde{h_t}$ is the candidate state, which is 
computed in this manner: 

\begin{equation*}
	\widetilde{h_t} = tanh(W_h x_t + r_t \cdot U_h h_{t-1} + b_h)
\end{equation*}

As mentioned previously, $r_t$ is the reset gate 
controlling how much of the information embedded in 
the previous state will contribute to the new candidate state. 
Note that if $r_t = 0$, the previous state is neglected. 
To update $r_t$, we have the following equation (similar to the update of $z_t$)

\begin{equation*}
	r_t = \sigma(W_r x_t + U_r h_{t-1} + b_r)
\end{equation*}

\paragraph*{Combining Attention}
Given an instance of the problem $T=\{seq_1, \ldots, seq_n\} \subseteq \mathcal{T}$, where $|seq_i| = k$ for some $k \in \mathbb{N}$, each sequence contains $S_i$ sentences, each sentence $s_i \in seq_j$ contains $T_i$ words. 
We describe the different levels of encoding the attention that lead 
to an encoding of the entire sequence. From there, we perform 
a binary classification. 

In the word level, we embed each word using an embedding matrix \cite{glove} 
and use a bidirectional GRU 
to get annotations of words by summarizing information
from both directions for words \cite{attention}. 
Employing both a forward and a backward GRU 
allows the encoder to get the contextual meaning of the words. 
For a given word $w_{it}, t\in [0,T]$, 
we concatenate the forward hidden state and
backward hidden state to get its annotation. 

In the attempt of distinguishing the important 
words in the sentence from the rest, we employ word-level 
attention. We highlight the words that most contribute 
to the meaning of the entire sentence so that we can represent 
a sentence vector $s_i$. 

First, the word annotation $h_{it}$ is fed 
through a one-layer MLP to get $u_{it}$ as a hidden representation
of $h_{it}$. Then, the normalized importance weight of the word $\alpha_{it}$ 
is computed as the similarity of $u_{it}$ with the word-level 
context vector $u_w$. 
Finally, the sentence vector $s_i$ is a weighted sum of the word annotations
based on the weights. 

\begin{equation*}
	u_{it} = tanh(W_w h_{it} + b_w)
\end{equation*}

\begin{equation*}
	\alpha_{it} = \frac{exp(u^T_{it} u_w)}{\sum_t exp(u^T_{it} u_w)}
\end{equation*}

\begin{equation*}
	s_i = \sum_{t} \alpha_{it} h_{it}
\end{equation*}

We use a similar mechanism (including the forward and backward 
GRU, and the attention layer) for the sentence-level vectors, 
for the sentences appearing in each turn in the sequence $seq_i$. 
The output of this stage is a turn vector. 

Once we have the representation of all the turns in $seq_i$, 
we again use a GRU to convert then into a sequence vector $v$. 

\subsection{Semantic Features}
The additional semantic features specified in Section \ref{sec:semantic} 
are added as coordinates in the sequence vector $v$, so that 
each vector also contains semantic features. 
All features are normalized by taking the largest 
value in each coordinate across all sequences and dividing 
all values in that coordinate by this value.  
Note that the length of the vector depends on $k$, 
i.e., the number of sentences in the sequences. 
A semantic vector has $k+2$ coordinates since feature \ref{itm:between} in Section \ref{sec:semantic} requires $k-1$ coordinates 
and there are $3$ additional features. 
% These vectors are the input for a fully-connected neural 
% network with 2 hidden layers, with 12 and 8 neurons, respectively. 
% As this is a binary classification, the network has an output layer of size 1. 
% For both hidden layers we have used the rectified linear unit as the activation 
% function and for the output we applied the sigmoid function.


\subsection{Classification}
After obtaining the abstract representation of the sequences, 
we use a layer of two neurons to classify the sequences and 
generate a function approximating the solution (Definition \ref{def:target}) 

\begin{equation*}
	p = softmax(W_c v + b_c)
\end{equation*}

Negative log-likelihood was used to correct labels ($j$ is the label of sequence $i$): 

\begin{equation*}
	L = - \sum_{i} \log{p_{ij}}
\end{equation*}

