%!TEX root = main.tex
\section{Introduction}\label{sec:into}
% \amir{reshape}
Dialog understanding has been the subject of 
extensive research \cite{BohusR03,BordesW16,GhazvininejadBC17,ShawarA03,DBLP:conf/icassp/JiB05,DBLP:conf/coling/WermterL96}. 
These works explore different aspects of dialogs, from winning an argument \cite{TanNDL16} 
to automatically recognize the emotions conveyed in a dialog at each segment \cite{AyadiKK11}. 
Namely, previous work \cite{ShawarA03,Jia09,ShawarA08,AngLS05,SurendranL06} has tried to learn dialog 
generation and tag dialog acts, towards building effective chatbots. 

Our work complements these efforts and aims to {\em understand the 
direction to which a goal-oriented dialog is headed using only its 
first part}. The motivation for this feature is to learn how 
well the conversation is going, {\em during its progress} in order 
to change strategy if necessary or end it. 
Uses for this feature include automatic chatbots whose goals 
are to sell a product by the end of the chat, a teacher and a student 
where the teacher tries to instruct the student how to achieve a certain task, 
and even to recognize a potential fraud email based on the first few 
messages.
Customer service teams can also benefit from the system as it can recognize 
whether a customer is calling to terminate the contract with the company 
or just to get a discount. 

The intuition behind our solution is that a dialog has verbal and 
non-verbal signs that may indicate its success or failure. 
In textual chats there are many ways to send non-verbal messages: 
timing of the responses, usage of icons, typos, misplaced punctuation marks etc.

Also in dialogs that are goal-oriented by nature, 
the topic of the dialog is crucial, and the way that 
the parties relate to the topic may be indicative to the success of the dialog. 
By using an ontology designed for the dialog type and subject, 
we can model important non-verbal aspects in the dialog.


% Also in dialogs that are goal-oriented by nature, the topic of the dialog is crucial, and the way that the parties relate to the topic may be indicative to the success of the conversation. There are knowledge bases that claim to capture the semantic distance between concepts (like ontologies). By using an ontology designed for the conversation type, and measure the semantic distance between the entities mentioned by the different parties in the conversation, we might model an important aspect in the conversation.

% For example two executives talk about a work process, each of them mentions different parts in the work process. Some stages are not very related, as research and deployment. Some are closer as deployment and testing. If one executive continues to talk about subjects related to research, while the other about deployment, then
% (given the right knowledge base) their corespondance will get high semantic distance, while if thay talked about testing and deploying (or about the same topic)
% it will receive low distance. 

We provide a formal model for the problem and a suggest 
different architectures of hierarchical neural networks 
that achieve promising results. 
We examine the performance of our architecture 
in multiple configurations on our benchmark dataset \cite{frames}. 


Our solution includes several stages. 
We first model the text in the given partial dialog and further extract semantic features from the dialogs. 
The language model we have used consists of three hierarchical layers \cite{attention}, the lower 
models the words, the second models the sentences, 
and the third novel layer models turns of each participant in the dialog. 
% and incorporated the semantic features in the turns layer. 
We have further incorporated attention \cite{BahdanauCB14} in each layer. 
% We then use an ontology to add yet more features that imply if the conversation 
% is headed towards its goal. 
We enrich our model with features taken from meta-data and an ontology. 
These features contain subtle observations about the dialog that 
would have remained obfuscated if the model would have only considered the dialog as plain text. 

% First, we will identify the topic of the conversation and named entities 
% appearing in it. Second, we will segment the conversation, 
% and cluster the different segments to identify types of 
% ``Conversational building blocks'' and learn probabilities 
% of transferring from one type of segment to another. 
% On top of the segments classification, we will extract features of 
% conversation (e.g., dominance of one of the parties, 
% the volume of text in the replies, usage of positive/negative words, 
% time lapses between responses). 
% We will use these features as input to MEMM/RNN to predict the 
% most probable continuation of the conversation 

% One of the challenges will be to identify the main topic of the whole conversation, and for each
% part of it. To address this issue our solution will use external knowledge base, for the relevant
% domain (i.e. if we will use e-commerce chatbots conversations - the knowledge base will be a
% catalog of products and their properties. For predicting outcome of geopolitical conversations we
% will use knowledge bases as YAGO and DBPedia). Using of the knowledge base will allow us to
% have better semantic understanding of the dialog, both while training the model and for testing. 
% Using knowledge base we can find similarities between different conversations (i.e. consider a chat about traveling suggestions, one about New York (on training step) and the second one about Tokyo (testing). Understand that both New York and Tokyo are cities will allow us to reuse the information about dialogs). Moreover, clustering dioalogs based on semantic similarity can be a boost for a better training.
